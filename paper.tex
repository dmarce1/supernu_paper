\documentclass[preprint]{aastex62}

\usepackage{amsmath}
\usepackage{natbib}
\bibliographystyle{apj}


\newcommand{\vdag}{(v)^\dagger}
\newcommand\aastex{AAS\TeX}
\newcommand\latex{La\TeX}
\newcommand{\pderiv}[2]{\frac{\partial}{\partial #2} #1}
\newcommand{\vect}[1]{{\bf #1}}
\newcommand{\diverg}[1]{ \vect{\nabla} \cdot #1 }
\newcommand{\eq}[1]{Equation (\ref{#1})}
\newcommand{\eqs}[2]{Equations (\ref{#1}) - (\ref{#2})}

\graphicspath{{./}{figures/}}

\shorttitle{SuperNU with Hydrodynamics}
\shortauthors{Marcello et. al.}

\begin{document}

\title{SuperNU with Hydrodynamics}


\author{Dominic C. Marcello}
\affiliation{Louisiana State University}
\author{Ryan T. Wollaeger}
\affiliation{Los Alamos National Laboratory}
\author{Wesley Even}
\affiliation{Los Alamos National Laboratory}
\author{Emmanouil Chatzopoulos}
\affiliation{Louisiana State University}
\begin{abstract}


\end{abstract}

\section{Introduction} \label{sec:intro}

\section{Model Equations} \label{sec:model}

The equations governing the hydrodynamic evolution of the system are
\begin{equation}
\label{rho_eq}
\pderiv{\rho_s}{t} + \diverg{\rho_s \vect{v}} + \frac{3 \rho_s}{t}  = 0,
\end{equation}
\begin{equation}
\label{mom_eq}
\pderiv{\rho \vect{u}}{t} + \diverg{\rho \vect{v}  \vect{u}}  + \frac{3 \rho  \vect{u}}{t}  = -\vect{g},
\end{equation}
\begin{equation}
\label{egas_eq}
\pderiv{E_\mathrm{gas}}{t} + \diverg{\left(\vect{v} E_\mathrm{gas} + \vect{u} P_\mathrm{gas}\right)}  + \frac{3 E_\mathrm{gas}}{t}  = -g^{\left(0\right)},
\end{equation}
and
\begin{equation}
\label{tau_eq}
\pderiv{\tau}{t} + \diverg{\tau \vect{v}} + \frac{3 \tau}{\gamma t}  = 0,
\end{equation}
$\rho_s$ is the mass density of each element, $\vect{g}$ is the momentum source due to radiation, and $g^{\left(0\right)}$ is the energy source due to radiation. The total density, $\rho$, is defined as the sum of all the elemental mass densities, 
\begin{equation}
\rho := \Sigma_s \rho_s.
\end{equation} 
The inertial frame velocity, $\vect{u}$, is defined relative to the lab frame fluid velocity, $\vect{v}$, according to
\begin{equation}
\vect{u} := \vect{v} + \frac{\vect{x}}{t},
\end{equation}
where $\vect{x}$ is the position vector relative to the origin. This defintion also requires that $t=0$ at the start of the homologous expansion. The total gas energy, $E_\mathrm{gas}$, is defined as the sum of the internal heat energy, $\rho \epsilon$, and the kinetic
energy,
\begin{equation}
E_\mathrm{gas} := \rho \epsilon + \tfrac{1}{2} \rho u^2.
\end{equation}
The gas pressure is set according to the ideal gas equation of state for a monatomic gas, 
\begin{equation}
P_\mathrm{gas} := \left(\gamma -1\right) \rho \epsilon,
\end{equation}
where $\gamma := \tfrac{5}{3}$. The entropy trace, $\tau$, is defined as 
\begin{equation}
\tau := \left( \rho \epsilon \right)^{\tfrac{1}{\gamma}}.
\end{equation}
The specific internal gas energy, $\epsilon$, can be determined from either $E_\mathrm{gas}$ or $\tau$. This choice is decided using the ``dual energy formalism'' \cite{ROKC1993}.

The hyperbolic parts of \eqs{rho_eq}{egas_eq} are solved using the semi-discrete scheme of \cite{KT2000}


\bibliography{paper}

\end{document}


